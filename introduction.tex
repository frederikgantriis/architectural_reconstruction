This report is written as part of the course "Software Architecture" which is active in the spring of 2025.

Architectural reconstruction is a crucial process in software engineering that enables developers and stakeholders to regain an understanding of a system’s structure, especially when documentation is missing, outdated, or incomplete. It supports maintainability, scalability, and informed decision-making by revealing how components interact and evolve over time.

This report focuses on the architectural reconstruction of the backend of Zeegu, a web-based platform with its backend implemented in Python. The goal is to reverse-engineer and document the system’s current architecture to provide a clearer overview of its structure and dependencies.
